% Options for packages loaded elsewhere
\PassOptionsToPackage{unicode}{hyperref}
\PassOptionsToPackage{hyphens}{url}
%
\documentclass[
]{article}
\usepackage{amsmath,amssymb}
\usepackage{lmodern}
\usepackage{iftex}
\ifPDFTeX
  \usepackage[T1]{fontenc}
  \usepackage[utf8]{inputenc}
  \usepackage{textcomp} % provide euro and other symbols
\else % if luatex or xetex
  \usepackage{unicode-math}
  \defaultfontfeatures{Scale=MatchLowercase}
  \defaultfontfeatures[\rmfamily]{Ligatures=TeX,Scale=1}
\fi
% Use upquote if available, for straight quotes in verbatim environments
\IfFileExists{upquote.sty}{\usepackage{upquote}}{}
\IfFileExists{microtype.sty}{% use microtype if available
  \usepackage[]{microtype}
  \UseMicrotypeSet[protrusion]{basicmath} % disable protrusion for tt fonts
}{}
\makeatletter
\@ifundefined{KOMAClassName}{% if non-KOMA class
  \IfFileExists{parskip.sty}{%
    \usepackage{parskip}
  }{% else
    \setlength{\parindent}{0pt}
    \setlength{\parskip}{6pt plus 2pt minus 1pt}}
}{% if KOMA class
  \KOMAoptions{parskip=half}}
\makeatother
\usepackage{xcolor}
\usepackage[margin=1in]{geometry}
\usepackage{color}
\usepackage{fancyvrb}
\newcommand{\VerbBar}{|}
\newcommand{\VERB}{\Verb[commandchars=\\\{\}]}
\DefineVerbatimEnvironment{Highlighting}{Verbatim}{commandchars=\\\{\}}
% Add ',fontsize=\small' for more characters per line
\usepackage{framed}
\definecolor{shadecolor}{RGB}{248,248,248}
\newenvironment{Shaded}{\begin{snugshade}}{\end{snugshade}}
\newcommand{\AlertTok}[1]{\textcolor[rgb]{0.94,0.16,0.16}{#1}}
\newcommand{\AnnotationTok}[1]{\textcolor[rgb]{0.56,0.35,0.01}{\textbf{\textit{#1}}}}
\newcommand{\AttributeTok}[1]{\textcolor[rgb]{0.77,0.63,0.00}{#1}}
\newcommand{\BaseNTok}[1]{\textcolor[rgb]{0.00,0.00,0.81}{#1}}
\newcommand{\BuiltInTok}[1]{#1}
\newcommand{\CharTok}[1]{\textcolor[rgb]{0.31,0.60,0.02}{#1}}
\newcommand{\CommentTok}[1]{\textcolor[rgb]{0.56,0.35,0.01}{\textit{#1}}}
\newcommand{\CommentVarTok}[1]{\textcolor[rgb]{0.56,0.35,0.01}{\textbf{\textit{#1}}}}
\newcommand{\ConstantTok}[1]{\textcolor[rgb]{0.00,0.00,0.00}{#1}}
\newcommand{\ControlFlowTok}[1]{\textcolor[rgb]{0.13,0.29,0.53}{\textbf{#1}}}
\newcommand{\DataTypeTok}[1]{\textcolor[rgb]{0.13,0.29,0.53}{#1}}
\newcommand{\DecValTok}[1]{\textcolor[rgb]{0.00,0.00,0.81}{#1}}
\newcommand{\DocumentationTok}[1]{\textcolor[rgb]{0.56,0.35,0.01}{\textbf{\textit{#1}}}}
\newcommand{\ErrorTok}[1]{\textcolor[rgb]{0.64,0.00,0.00}{\textbf{#1}}}
\newcommand{\ExtensionTok}[1]{#1}
\newcommand{\FloatTok}[1]{\textcolor[rgb]{0.00,0.00,0.81}{#1}}
\newcommand{\FunctionTok}[1]{\textcolor[rgb]{0.00,0.00,0.00}{#1}}
\newcommand{\ImportTok}[1]{#1}
\newcommand{\InformationTok}[1]{\textcolor[rgb]{0.56,0.35,0.01}{\textbf{\textit{#1}}}}
\newcommand{\KeywordTok}[1]{\textcolor[rgb]{0.13,0.29,0.53}{\textbf{#1}}}
\newcommand{\NormalTok}[1]{#1}
\newcommand{\OperatorTok}[1]{\textcolor[rgb]{0.81,0.36,0.00}{\textbf{#1}}}
\newcommand{\OtherTok}[1]{\textcolor[rgb]{0.56,0.35,0.01}{#1}}
\newcommand{\PreprocessorTok}[1]{\textcolor[rgb]{0.56,0.35,0.01}{\textit{#1}}}
\newcommand{\RegionMarkerTok}[1]{#1}
\newcommand{\SpecialCharTok}[1]{\textcolor[rgb]{0.00,0.00,0.00}{#1}}
\newcommand{\SpecialStringTok}[1]{\textcolor[rgb]{0.31,0.60,0.02}{#1}}
\newcommand{\StringTok}[1]{\textcolor[rgb]{0.31,0.60,0.02}{#1}}
\newcommand{\VariableTok}[1]{\textcolor[rgb]{0.00,0.00,0.00}{#1}}
\newcommand{\VerbatimStringTok}[1]{\textcolor[rgb]{0.31,0.60,0.02}{#1}}
\newcommand{\WarningTok}[1]{\textcolor[rgb]{0.56,0.35,0.01}{\textbf{\textit{#1}}}}
\usepackage{graphicx}
\makeatletter
\def\maxwidth{\ifdim\Gin@nat@width>\linewidth\linewidth\else\Gin@nat@width\fi}
\def\maxheight{\ifdim\Gin@nat@height>\textheight\textheight\else\Gin@nat@height\fi}
\makeatother
% Scale images if necessary, so that they will not overflow the page
% margins by default, and it is still possible to overwrite the defaults
% using explicit options in \includegraphics[width, height, ...]{}
\setkeys{Gin}{width=\maxwidth,height=\maxheight,keepaspectratio}
% Set default figure placement to htbp
\makeatletter
\def\fps@figure{htbp}
\makeatother
\setlength{\emergencystretch}{3em} % prevent overfull lines
\providecommand{\tightlist}{%
  \setlength{\itemsep}{0pt}\setlength{\parskip}{0pt}}
\setcounter{secnumdepth}{-\maxdimen} % remove section numbering
\ifLuaTeX
  \usepackage{selnolig}  % disable illegal ligatures
\fi
\IfFileExists{bookmark.sty}{\usepackage{bookmark}}{\usepackage{hyperref}}
\IfFileExists{xurl.sty}{\usepackage{xurl}}{} % add URL line breaks if available
\urlstyle{same} % disable monospaced font for URLs
\hypersetup{
  pdftitle={Homework 4},
  pdfauthor={Zejie Gao},
  hidelinks,
  pdfcreator={LaTeX via pandoc}}

\title{Homework 4}
\author{Zejie Gao}
\date{Due date:March 17th 2023 at 23:59 PT}

\begin{document}
\maketitle

\begin{enumerate}
\def\labelenumi{\arabic{enumi}.}
\tightlist
\item
  This question uses the Auto dataset available in the ISLR package. The
  dataset under the name \emph{Auto} is automatically available once the
  ISLR package is loaded.
\end{enumerate}

\begin{Shaded}
\begin{Highlighting}[]
\FunctionTok{library}\NormalTok{(ISLR)}
\FunctionTok{data}\NormalTok{(Auto)}
\FunctionTok{library}\NormalTok{(}\StringTok{"tidyverse"}\NormalTok{)}
\end{Highlighting}
\end{Shaded}

\begin{verbatim}
## -- Attaching packages --------------------------------------- tidyverse 1.3.2 --
## v ggplot2 3.3.6     v purrr   0.3.4
## v tibble  3.1.8     v dplyr   1.1.0
## v tidyr   1.2.0     v stringr 1.4.0
## v readr   2.1.2     v forcats 0.5.1
## -- Conflicts ------------------------------------------ tidyverse_conflicts() --
## x dplyr::filter() masks stats::filter()
## x dplyr::lag()    masks stats::lag()
\end{verbatim}

\begin{Shaded}
\begin{Highlighting}[]
\FunctionTok{library}\NormalTok{(}\StringTok{"dplyr"}\NormalTok{)}
\FunctionTok{library}\NormalTok{(}\StringTok{"lmtest"}\NormalTok{)}
\end{Highlighting}
\end{Shaded}

\begin{verbatim}
## 载入需要的程辑包:zoo
## 
## 载入程辑包:'zoo'
## 
## The following objects are masked from 'package:base':
## 
##     as.Date, as.Date.numeric
\end{verbatim}

\begin{Shaded}
\begin{Highlighting}[]
\FunctionTok{library}\NormalTok{(}\StringTok{"MASS"}\NormalTok{)}
\end{Highlighting}
\end{Shaded}

\begin{verbatim}
## 
## 载入程辑包:'MASS'
## 
## The following object is masked from 'package:dplyr':
## 
##     select
\end{verbatim}

The dataset \emph{Auto} contains the following information for \(392\)
vehicles:

\begin{itemize}
\tightlist
\item
  mpg: miles per gallon
\item
  cylinders: number of cylinders (between 4 and 8)
\item
  displacement: engine displacement (cu.inches)
\item
  horsepower: engine horsepower
\item
  weight: vehicle weight (lbs)
\item
  acceleration: time to accelerate from 0 to 60 mph (seconds)
\item
  year: model year
\item
  origin: origin of the vehicle (numerically coded as 1: American, 2:
  European, 3: Japanese)
\item
  name: vehicle name
\end{itemize}

Our goal is to analyze several linear models where \emph{mpg} is the
response variable.\\

\begin{enumerate}
\def\labelenumi{(\alph{enumi})}
\tightlist
\item
  \textbf{(2 pts)} In this data set, which predictors are qualitative,
  and which predictors are quantitative? In this data set, mpg,
  displacement, horsepower, weight and acceleration are quantitative,
  and the rest of the predictors such as cylinders, year and origin are
  qualitative.
\end{enumerate}

\begin{Shaded}
\begin{Highlighting}[]
\FunctionTok{summary}\NormalTok{(Autod)}
\end{Highlighting}
\end{Shaded}

\begin{verbatim}
##       mpg        cylinders  displacement     horsepower        weight    
##  Min.   : 9.00   3:  4     Min.   : 68.0   Min.   : 46.0   Min.   :1613  
##  1st Qu.:17.00   4:199     1st Qu.:105.0   1st Qu.: 75.0   1st Qu.:2225  
##  Median :22.75   5:  3     Median :151.0   Median : 93.5   Median :2804  
##  Mean   :23.45   6: 83     Mean   :194.4   Mean   :104.5   Mean   :2978  
##  3rd Qu.:29.00   8:103     3rd Qu.:275.8   3rd Qu.:126.0   3rd Qu.:3615  
##  Max.   :46.60             Max.   :455.0   Max.   :230.0   Max.   :5140  
##                                                                          
##   acceleration        year     origin                  name    
##  Min.   : 8.00   73     : 40   1:245   amc matador       :  5  
##  1st Qu.:13.78   78     : 36   2: 68   ford pinto        :  5  
##  Median :15.50   76     : 34   3: 79   toyota corolla    :  5  
##  Mean   :15.54   75     : 30           amc gremlin       :  4  
##  3rd Qu.:17.02   82     : 30           amc hornet        :  4  
##  Max.   :24.80   70     : 29           chevrolet chevette:  4  
##                  (Other):193           (Other)           :365
\end{verbatim}

\begin{enumerate}
\def\labelenumi{(\alph{enumi})}
\setcounter{enumi}{1}
\tightlist
\item
  \textbf{(2 pts)} Fit a MLR model to the data, in order to predict mpg
  using all of the other predictors except for name. For each predictor
  in the fitted MLR model, comment on whether you can reject the null
  hypothesis that there is no linear association between that predictor
  and mpg, conditional on the other predictors in the model. Looking at
  the analysis of summary table, we see that all of the predictors
  except for acceleration and displacement have a very low p-value (less
  than 0.05), indicating strong evidence that there is a linear
  association between each of these predictors and mpg, conditional on
  the other predictors in the model. As acceleration, the p-value
  (0.3315) is greater than 0.05, suggesting that fail to reject the null
  hypothesis that there is no linear association between between
  acceleration and mpg, after controlling for the other predictors in
  the model. As displacement, the p-value (0.081785) is silgtly grater
  than 0.05; thus, they don't have linear association when using 5\%
  significant level. Although there are variables within the predictor
  ``year'' (specifically, year71 and year72) that are not statistically
  significant, it is still reasonable to consider ``year'' as a
  predictor of the outcome variable due to the presence of other
  variables within the predictor that do show statistical significance
  (namely, year77 and year78).
\end{enumerate}

\begin{Shaded}
\begin{Highlighting}[]
\NormalTok{lmod}\OtherTok{\textless{}{-}} \FunctionTok{lm}\NormalTok{(mpg}\SpecialCharTok{\textasciitilde{}}\NormalTok{ cylinders }\SpecialCharTok{+}\NormalTok{ displacement }\SpecialCharTok{+}\NormalTok{ horsepower }\SpecialCharTok{+}\NormalTok{ weight }\SpecialCharTok{+}\NormalTok{ acceleration }\SpecialCharTok{+}
\NormalTok{                       year }\SpecialCharTok{+}\NormalTok{ origin, Autod)}
\FunctionTok{summary}\NormalTok{(lmod)}
\end{Highlighting}
\end{Shaded}

\begin{verbatim}
## 
## Call:
## lm(formula = mpg ~ cylinders + displacement + horsepower + weight + 
##     acceleration + year + origin, data = Autod)
## 
## Residuals:
##     Min      1Q  Median      3Q     Max 
## -7.9267 -1.6678 -0.0506  1.4493 11.6002 
## 
## Coefficients:
##                Estimate Std. Error t value Pr(>|t|)    
## (Intercept)  30.9168415  2.3608985  13.095  < 2e-16 ***
## cylinders4    6.9399216  1.5365961   4.516 8.48e-06 ***
## cylinders5    6.6377310  2.3372687   2.840 0.004762 ** 
## cylinders6    4.2973139  1.7057848   2.519 0.012182 *  
## cylinders8    6.3668129  1.9687277   3.234 0.001331 ** 
## displacement  0.0118246  0.0067755   1.745 0.081785 .  
## horsepower   -0.0392323  0.0130356  -3.010 0.002795 ** 
## weight       -0.0051802  0.0006241  -8.300 1.99e-15 ***
## acceleration  0.0036080  0.0868925   0.042 0.966902    
## year71        0.9104285  0.8155744   1.116 0.265019    
## year72       -0.4903062  0.8038193  -0.610 0.542257    
## year73       -0.5528934  0.7214463  -0.766 0.443947    
## year74        1.2419976  0.8547434   1.453 0.147056    
## year75        0.8704016  0.8374036   1.039 0.299297    
## year76        1.4966598  0.8019080   1.866 0.062782 .  
## year77        2.9986967  0.8198949   3.657 0.000292 ***
## year78        2.9737783  0.7792185   3.816 0.000159 ***
## year79        4.8961763  0.8248124   5.936 6.74e-09 ***
## year80        9.0589316  0.8751948  10.351  < 2e-16 ***
## year81        6.4581580  0.8637018   7.477 5.58e-13 ***
## year82        7.8375850  0.8493560   9.228  < 2e-16 ***
## origin2       1.6932853  0.5162117   3.280 0.001136 ** 
## origin3       2.2929268  0.4967645   4.616 5.41e-06 ***
## ---
## Signif. codes:  0 '***' 0.001 '**' 0.01 '*' 0.05 '.' 0.1 ' ' 1
## 
## Residual standard error: 2.848 on 369 degrees of freedom
## Multiple R-squared:  0.8744, Adjusted R-squared:  0.8669 
## F-statistic: 116.8 on 22 and 369 DF,  p-value: < 2.2e-16
\end{verbatim}

\begin{enumerate}
\def\labelenumi{(\alph{enumi})}
\setcounter{enumi}{2}
\tightlist
\item
  \textbf{(2 pts)} What mpg do you predict for a Japanese car with three
  cylinders, displacement 100, horsepower of 85, weight of 3000,
  acceleration of 20, built in the year 1980?
\end{enumerate}

\begin{Shaded}
\begin{Highlighting}[]
\NormalTok{new\_data }\OtherTok{\textless{}{-}} \FunctionTok{data.frame}\NormalTok{(}\AttributeTok{cylinders =} \FunctionTok{factor}\NormalTok{(}\DecValTok{3}\NormalTok{, }\AttributeTok{levels =} \FunctionTok{levels}\NormalTok{(Autod}\SpecialCharTok{$}\NormalTok{cylinders)),}
                       \AttributeTok{displacement =} \DecValTok{100}\NormalTok{, }
                       \AttributeTok{horsepower =} \DecValTok{85}\NormalTok{, }
                       \AttributeTok{weight =} \DecValTok{3000}\NormalTok{, }
                       \AttributeTok{acceleration =} \DecValTok{20}\NormalTok{, }
                       \AttributeTok{year =} \FunctionTok{factor}\NormalTok{(}\DecValTok{80}\NormalTok{, }\AttributeTok{levels =} \FunctionTok{levels}\NormalTok{(Autod}\SpecialCharTok{$}\NormalTok{year)),}
                       \AttributeTok{origin =} \FunctionTok{factor}\NormalTok{(}\DecValTok{3}\NormalTok{, }\AttributeTok{levels =} \FunctionTok{levels}\NormalTok{(Autod}\SpecialCharTok{$}\NormalTok{origin)))}
\NormalTok{predicted\_mpg }\OtherTok{\textless{}{-}} \FunctionTok{predict}\NormalTok{(lmod, }\AttributeTok{newdata =}\NormalTok{ new\_data, }\AttributeTok{interval =} \StringTok{"prediction"}\NormalTok{)}
\NormalTok{predicted\_mpg}
\end{Highlighting}
\end{Shaded}

\begin{verbatim}
##        fit      lwr      upr
## 1 24.64804 18.24614 31.04993
\end{verbatim}

\begin{enumerate}
\def\labelenumi{(\alph{enumi})}
\setcounter{enumi}{3}
\tightlist
\item
  \textbf{(2 pts)} On average, holding all other predictor variables
  fixed, what is the difference between the mpg of a Japanese car and
  the mpg of an European car? Therefore, on average, holding all other
  predictor variables fixed, the mpg of a Japanese car is 0.5996415
  higher than the mpg of an European car.
\end{enumerate}

\begin{Shaded}
\begin{Highlighting}[]
\FunctionTok{summary}\NormalTok{(lmod)}
\end{Highlighting}
\end{Shaded}

\begin{verbatim}
## 
## Call:
## lm(formula = mpg ~ cylinders + displacement + horsepower + weight + 
##     acceleration + year + origin, data = Autod)
## 
## Residuals:
##     Min      1Q  Median      3Q     Max 
## -7.9267 -1.6678 -0.0506  1.4493 11.6002 
## 
## Coefficients:
##                Estimate Std. Error t value Pr(>|t|)    
## (Intercept)  30.9168415  2.3608985  13.095  < 2e-16 ***
## cylinders4    6.9399216  1.5365961   4.516 8.48e-06 ***
## cylinders5    6.6377310  2.3372687   2.840 0.004762 ** 
## cylinders6    4.2973139  1.7057848   2.519 0.012182 *  
## cylinders8    6.3668129  1.9687277   3.234 0.001331 ** 
## displacement  0.0118246  0.0067755   1.745 0.081785 .  
## horsepower   -0.0392323  0.0130356  -3.010 0.002795 ** 
## weight       -0.0051802  0.0006241  -8.300 1.99e-15 ***
## acceleration  0.0036080  0.0868925   0.042 0.966902    
## year71        0.9104285  0.8155744   1.116 0.265019    
## year72       -0.4903062  0.8038193  -0.610 0.542257    
## year73       -0.5528934  0.7214463  -0.766 0.443947    
## year74        1.2419976  0.8547434   1.453 0.147056    
## year75        0.8704016  0.8374036   1.039 0.299297    
## year76        1.4966598  0.8019080   1.866 0.062782 .  
## year77        2.9986967  0.8198949   3.657 0.000292 ***
## year78        2.9737783  0.7792185   3.816 0.000159 ***
## year79        4.8961763  0.8248124   5.936 6.74e-09 ***
## year80        9.0589316  0.8751948  10.351  < 2e-16 ***
## year81        6.4581580  0.8637018   7.477 5.58e-13 ***
## year82        7.8375850  0.8493560   9.228  < 2e-16 ***
## origin2       1.6932853  0.5162117   3.280 0.001136 ** 
## origin3       2.2929268  0.4967645   4.616 5.41e-06 ***
## ---
## Signif. codes:  0 '***' 0.001 '**' 0.01 '*' 0.05 '.' 0.1 ' ' 1
## 
## Residual standard error: 2.848 on 369 degrees of freedom
## Multiple R-squared:  0.8744, Adjusted R-squared:  0.8669 
## F-statistic: 116.8 on 22 and 369 DF,  p-value: < 2.2e-16
\end{verbatim}

\begin{Shaded}
\begin{Highlighting}[]
\NormalTok{dif\_mpg\_J\_E }\OtherTok{\textless{}{-}} \FloatTok{2.2929268{-}1.6932853}\NormalTok{; dif\_mpg\_J\_E}
\end{Highlighting}
\end{Shaded}

\begin{verbatim}
## [1] 0.5996415
\end{verbatim}

\begin{enumerate}
\def\labelenumi{(\alph{enumi})}
\setcounter{enumi}{4}
\tightlist
\item
  \textbf{(2 pts)} Fit a model to predict \emph{mpg} using origin and
  horsepower, as well as an interaction between origin and horsepower.
  Present the summary output of the fitted model, and write out the
  fitted linear model.
  \[\hat{\text{mpg}}=34.476496-0.121320*\text{horsepower}+10.99723*I(\text{origin=2})+14.339718*I(\text{origin=3})-0.100515*\text{horsepower}*I(\text{origin=2})-0.108723*\text{horsepower}*I(\text{origin=3}).\]
\end{enumerate}

\begin{Shaded}
\begin{Highlighting}[]
\NormalTok{mod2 }\OtherTok{\textless{}{-}}\FunctionTok{lm}\NormalTok{(mpg}\SpecialCharTok{\textasciitilde{}}\NormalTok{ horsepower }\SpecialCharTok{+}\NormalTok{ origin }\SpecialCharTok{+}\NormalTok{ horsepower}\SpecialCharTok{:}\NormalTok{origin, Autod)}
\FunctionTok{summary}\NormalTok{(mod2)}
\end{Highlighting}
\end{Shaded}

\begin{verbatim}
## 
## Call:
## lm(formula = mpg ~ horsepower + origin + horsepower:origin, data = Autod)
## 
## Residuals:
##      Min       1Q   Median       3Q      Max 
## -10.7415  -2.9547  -0.6389   2.3978  14.2495 
## 
## Coefficients:
##                     Estimate Std. Error t value Pr(>|t|)    
## (Intercept)        34.476496   0.890665  38.709  < 2e-16 ***
## horsepower         -0.121320   0.007095 -17.099  < 2e-16 ***
## origin2            10.997230   2.396209   4.589 6.02e-06 ***
## origin3            14.339718   2.464293   5.819 1.24e-08 ***
## horsepower:origin2 -0.100515   0.027723  -3.626 0.000327 ***
## horsepower:origin3 -0.108723   0.028980  -3.752 0.000203 ***
## ---
## Signif. codes:  0 '***' 0.001 '**' 0.01 '*' 0.05 '.' 0.1 ' ' 1
## 
## Residual standard error: 4.422 on 386 degrees of freedom
## Multiple R-squared:  0.6831, Adjusted R-squared:  0.679 
## F-statistic: 166.4 on 5 and 386 DF,  p-value: < 2.2e-16
\end{verbatim}

\begin{enumerate}
\def\labelenumi{(\alph{enumi})}
\setcounter{enumi}{5}
\tightlist
\item
  \textbf{(2 pts)} If we are fitting a polynomial regression with mpg as
  the response variable and weight as the predictor, what should be a
  proper degree of that polynomial? The p-values in each model's output
  indicate whether each predictor variable's coefficient is
  significantly different from zero. A p-value less than 0.05 suggests
  strong evidence against the null hypothesis that the coefficient is
  equal to zero, and we can conclude that the predictor variable is
  significantly associated with the response variable. From there model
  below, only model 3 have p-value that is bigger than 0.05, suggesting
  that weight\^{}3 is a significant predictor of mpg in m3. Additional,
  the residual vs fitted value plot in m2 is more flatter than that in
  m1. Thus, second should be a proper degree of that polynomial,
  quadratic models.
\end{enumerate}

\begin{Shaded}
\begin{Highlighting}[]
\FunctionTok{summary}\NormalTok{(m1 }\OtherTok{\textless{}{-}} \FunctionTok{lm}\NormalTok{(mpg}\SpecialCharTok{\textasciitilde{}}\NormalTok{weight,Autod))}\SpecialCharTok{$}\NormalTok{coefficient}
\end{Highlighting}
\end{Shaded}

\begin{verbatim}
##                 Estimate   Std. Error   t value      Pr(>|t|)
## (Intercept) 46.216524549 0.7986724633  57.86668 1.623069e-193
## weight      -0.007647343 0.0002579633 -29.64508 6.015296e-102
\end{verbatim}

\begin{Shaded}
\begin{Highlighting}[]
\FunctionTok{summary}\NormalTok{(m2 }\OtherTok{\textless{}{-}} \FunctionTok{lm}\NormalTok{(mpg}\SpecialCharTok{\textasciitilde{}}\NormalTok{weight }\SpecialCharTok{+} \FunctionTok{I}\NormalTok{(weight}\SpecialCharTok{\^{}}\DecValTok{2}\NormalTok{),Autod))}\SpecialCharTok{$}\NormalTok{coefficient}
\end{Highlighting}
\end{Shaded}

\begin{verbatim}
##                  Estimate   Std. Error   t value     Pr(>|t|)
## (Intercept)  6.225547e+01 2.993076e+00 20.799832 3.848779e-65
## weight      -1.849561e-02 1.972056e-03 -9.378849 5.609944e-19
## I(weight^2)  1.696565e-06 3.059491e-07  5.545252 5.429177e-08
\end{verbatim}

\begin{Shaded}
\begin{Highlighting}[]
\FunctionTok{summary}\NormalTok{(m3 }\OtherTok{\textless{}{-}} \FunctionTok{lm}\NormalTok{(mpg}\SpecialCharTok{\textasciitilde{}}\NormalTok{weight }\SpecialCharTok{+} \FunctionTok{I}\NormalTok{(weight}\SpecialCharTok{\^{}}\DecValTok{2}\NormalTok{) }\SpecialCharTok{+} \FunctionTok{I}\NormalTok{(weight}\SpecialCharTok{\^{}}\DecValTok{3}\NormalTok{),Autod))}\SpecialCharTok{$}\NormalTok{coefficient}
\end{Highlighting}
\end{Shaded}

\begin{verbatim}
##                  Estimate   Std. Error     t value     Pr(>|t|)
## (Intercept)  6.169524e+01 1.104305e+01  5.58679434 4.360869e-08
## weight      -1.792978e-02 1.091485e-02 -1.64269604 1.012560e-01
## I(weight^2)  1.515412e-06 3.450428e-06  0.43919548 6.607644e-01
## I(weight^3)  1.846219e-11 3.502615e-10  0.05270974 9.579903e-01
\end{verbatim}

\begin{Shaded}
\begin{Highlighting}[]
\FunctionTok{par}\NormalTok{(}\AttributeTok{mfrow =} \FunctionTok{c}\NormalTok{(}\DecValTok{1}\NormalTok{, }\DecValTok{4}\NormalTok{), }\AttributeTok{mar =} \FunctionTok{c}\NormalTok{(}\DecValTok{0}\NormalTok{,}\DecValTok{0}\NormalTok{,}\FloatTok{1.5}\NormalTok{,}\DecValTok{1}\NormalTok{))}
\FunctionTok{plot}\NormalTok{(m1, }\AttributeTok{cex.main =} \DecValTok{1}\NormalTok{, }\AttributeTok{cex.lab =} \FloatTok{0.5}\NormalTok{, }\AttributeTok{cex.axis =} \FloatTok{0.5}\NormalTok{, }\AttributeTok{pch =} \DecValTok{20}\NormalTok{)}
\end{Highlighting}
\end{Shaded}

\includegraphics{Hw4_126_files/figure-latex/unnamed-chunk-6-1.pdf}

\begin{Shaded}
\begin{Highlighting}[]
\FunctionTok{plot}\NormalTok{(m2, }\AttributeTok{cex.main =} \DecValTok{1}\NormalTok{, }\AttributeTok{cex.lab =} \FloatTok{0.5}\NormalTok{, }\AttributeTok{cex.axis =} \FloatTok{0.5}\NormalTok{, }\AttributeTok{pch =} \DecValTok{20}\NormalTok{)}
\end{Highlighting}
\end{Shaded}

\includegraphics{Hw4_126_files/figure-latex/unnamed-chunk-6-2.pdf}

\begin{enumerate}
\def\labelenumi{(\alph{enumi})}
\setcounter{enumi}{6}
\tightlist
\item
  \textbf{(4 pts)} Perform a backward selection, starting with the full
  model which includes all predictors (except for name). What is the
  best model based on the AIC criterion? What are the predictor
  variables in that best model?
\end{enumerate}

The AIC value will decrease as the model fits the data better.In the
first step, the acceleration variable is removed from the model,
resulting in a lower AIC value of 840.72. This means that the model
without acceleration is a better fit for the data than the original
model.In the second step, the remaining predictor variables are
cylinders, displacement, horsepower, weight, year, and origin. The
output shows that no other variables should be removed from the model
since the AIC value remains the same as before. Therefore, this is the
best model based on AIC criterion. The AIC values indicate that
cylinders, displacement, horsepower, weight, year, and origin are the
predictor variables in that best model. formula = mpg \textasciitilde{}
cylinders + displacement + horsepower + weight + year + origin

\begin{Shaded}
\begin{Highlighting}[]
\FunctionTok{step}\NormalTok{(lmod, }\AttributeTok{direction =} \StringTok{"backward"}\NormalTok{)}
\end{Highlighting}
\end{Shaded}

\begin{verbatim}
## Start:  AIC=842.72
## mpg ~ cylinders + displacement + horsepower + weight + acceleration + 
##     year + origin
## 
##                Df Sum of Sq    RSS     AIC
## - acceleration  1      0.01 2992.1  840.72
## <none>                      2992.1  842.72
## - displacement  1     24.70 3016.8  843.94
## - horsepower    1     73.45 3065.5  850.23
## - origin        2    183.21 3175.3  862.02
## - cylinders     4    472.77 3464.8  892.23
## - weight        1    558.60 3550.7  907.82
## - year         12   2831.60 5823.7 1079.78
## 
## Step:  AIC=840.72
## mpg ~ cylinders + displacement + horsepower + weight + year + 
##     origin
## 
##                Df Sum of Sq    RSS     AIC
## <none>                      2992.1  840.72
## - displacement  1     24.88 3017.0  841.97
## - horsepower    1    115.58 3107.7  853.58
## - origin        2    183.45 3175.5  860.05
## - cylinders     4    476.39 3468.5  890.64
## - weight        1    730.02 3722.1  924.31
## - year         12   2841.52 5833.6 1078.45
\end{verbatim}

\begin{verbatim}
## 
## Call:
## lm(formula = mpg ~ cylinders + displacement + horsepower + weight + 
##     year + origin, data = Autod)
## 
## Coefficients:
##  (Intercept)    cylinders4    cylinders5    cylinders6    cylinders8  
##    30.970678      6.948983      6.646736      4.305068      6.372326  
## displacement    horsepower        weight        year71        year72  
##     0.011793     -0.039554     -0.005168      0.905750     -0.492137  
##       year73        year74        year75        year76        year77  
##    -0.555066      1.237612      0.865415      1.492399      2.994879  
##       year78        year79        year80        year81        year82  
##     2.970303      4.892261      9.055269      6.452705      7.833655  
##      origin2       origin3  
##     1.693186      2.293670
\end{verbatim}

\begin{enumerate}
\def\labelenumi{\arabic{enumi}.}
\setcounter{enumi}{1}
\tightlist
\item
  Use the \emph{fat} data set available from the \emph{faraway} package.
  Use the percentage of body fat: \emph{siri} as the response, and the
  other variables, except \emph{bronzek} and \emph{density} as potential
  predictors. Remove every tenth observation from the data for use as a
  test sample. Use the remaining data as a training sample, building the
  following models:
\end{enumerate}

\begin{Shaded}
\begin{Highlighting}[]
\FunctionTok{library}\NormalTok{(faraway)}
\FunctionTok{data}\NormalTok{(fat)}
\FunctionTok{head}\NormalTok{(fat)}
\end{Highlighting}
\end{Shaded}

\begin{verbatim}
##   brozek siri density age weight height adipos  free neck chest abdom   hip
## 1   12.6 12.3  1.0708  23 154.25  67.75   23.7 134.9 36.2  93.1  85.2  94.5
## 2    6.9  6.1  1.0853  22 173.25  72.25   23.4 161.3 38.5  93.6  83.0  98.7
## 3   24.6 25.3  1.0414  22 154.00  66.25   24.7 116.0 34.0  95.8  87.9  99.2
## 4   10.9 10.4  1.0751  26 184.75  72.25   24.9 164.7 37.4 101.8  86.4 101.2
## 5   27.8 28.7  1.0340  24 184.25  71.25   25.6 133.1 34.4  97.3 100.0 101.9
## 6   20.6 20.9  1.0502  24 210.25  74.75   26.5 167.0 39.0 104.5  94.4 107.8
##   thigh knee ankle biceps forearm wrist
## 1  59.0 37.3  21.9   32.0    27.4  17.1
## 2  58.7 37.3  23.4   30.5    28.9  18.2
## 3  59.6 38.9  24.0   28.8    25.2  16.6
## 4  60.1 37.3  22.8   32.4    29.4  18.2
## 5  63.2 42.2  24.0   32.2    27.7  17.7
## 6  66.0 42.0  25.6   35.7    30.6  18.8
\end{verbatim}

\begin{Shaded}
\begin{Highlighting}[]
\NormalTok{fat }\OtherTok{\textless{}{-}} \FunctionTok{subset}\NormalTok{(fat, }\AttributeTok{select =} \FunctionTok{c}\NormalTok{(}\DecValTok{2}\NormalTok{, }\DecValTok{4}\SpecialCharTok{:}\DecValTok{18}\NormalTok{))}
\NormalTok{test\_indices }\OtherTok{\textless{}{-}} \FunctionTok{seq}\NormalTok{(}\DecValTok{10}\NormalTok{, }\FunctionTok{nrow}\NormalTok{(fat), }\AttributeTok{by=}\DecValTok{10}\NormalTok{)}
\NormalTok{test\_data }\OtherTok{\textless{}{-}}\NormalTok{ fat[test\_indices,] }
\NormalTok{training\_data }\OtherTok{\textless{}{-}}\NormalTok{ fat[}\SpecialCharTok{{-}}\NormalTok{test\_indices, ]}
\end{Highlighting}
\end{Shaded}

\begin{enumerate}
\def\labelenumi{(\alph{enumi})}
\tightlist
\item
  \textbf{(5 pts)} Linear regression with all the predictors.
\end{enumerate}

\begin{Shaded}
\begin{Highlighting}[]
\NormalTok{MLR\_f }\OtherTok{\textless{}{-}} \FunctionTok{lm}\NormalTok{(siri}\SpecialCharTok{\textasciitilde{}}\NormalTok{.,training\_data);}\FunctionTok{summary}\NormalTok{(MLR\_f)}
\end{Highlighting}
\end{Shaded}

\begin{verbatim}
## 
## Call:
## lm(formula = siri ~ ., data = training_data)
## 
## Residuals:
##     Min      1Q  Median      3Q     Max 
## -5.8314 -0.6722  0.1828  0.9150  6.6619 
## 
## Coefficients:
##               Estimate Std. Error t value Pr(>|t|)    
## (Intercept) -12.591885   6.448868  -1.953 0.052193 .  
## age           0.007978   0.012320   0.648 0.517983    
## weight        0.362999   0.023314  15.570  < 2e-16 ***
## height        0.049026   0.040315   1.216 0.225315    
## adipos       -0.514032   0.114074  -4.506 1.09e-05 ***
## free         -0.564773   0.014889 -37.933  < 2e-16 ***
## neck          0.016525   0.089863   0.184 0.854272    
## chest         0.120219   0.039590   3.037 0.002694 ** 
## abdom         0.140108   0.042186   3.321 0.001056 ** 
## hip           0.006197   0.056101   0.110 0.912148    
## thigh         0.195057   0.054460   3.582 0.000424 ***
## knee          0.106637   0.093534   1.140 0.255542    
## ankle         0.125118   0.081303   1.539 0.125325    
## biceps        0.096199   0.064656   1.488 0.138278    
## forearm       0.230775   0.073332   3.147 0.001888 ** 
## wrist         0.139279   0.206804   0.673 0.501378    
## ---
## Signif. codes:  0 '***' 0.001 '**' 0.01 '*' 0.05 '.' 0.1 ' ' 1
## 
## Residual standard error: 1.55 on 211 degrees of freedom
## Multiple R-squared:  0.9692, Adjusted R-squared:  0.967 
## F-statistic: 442.5 on 15 and 211 DF,  p-value: < 2.2e-16
\end{verbatim}

\begin{enumerate}
\def\labelenumi{(\alph{enumi})}
\setcounter{enumi}{1}
\tightlist
\item
  \textbf{(5 pts)} Ridge regression.
\end{enumerate}

\begin{Shaded}
\begin{Highlighting}[]
\FunctionTok{library}\NormalTok{(glmnet)}
\end{Highlighting}
\end{Shaded}

\begin{verbatim}
## 载入需要的程辑包:Matrix
\end{verbatim}

\begin{verbatim}
## 
## 载入程辑包:'Matrix'
\end{verbatim}

\begin{verbatim}
## The following objects are masked from 'package:tidyr':
## 
##     expand, pack, unpack
\end{verbatim}

\begin{verbatim}
## Loaded glmnet 4.1-6
\end{verbatim}

\begin{Shaded}
\begin{Highlighting}[]
\NormalTok{x }\OtherTok{\textless{}{-}} \FunctionTok{scale}\NormalTok{(}\FunctionTok{data.matrix}\NormalTok{(fat)[,}\SpecialCharTok{{-}}\DecValTok{1}\NormalTok{])}
\NormalTok{x }\OtherTok{\textless{}{-}}\NormalTok{ x[}\SpecialCharTok{{-}}\NormalTok{test\_indices,]}
\NormalTok{y }\OtherTok{\textless{}{-}}\NormalTok{ training\_data}\SpecialCharTok{$}\NormalTok{siri}
\NormalTok{ridge\_model }\OtherTok{\textless{}{-}} \FunctionTok{cv.glmnet}\NormalTok{(x, y, }\AttributeTok{alpha =} \DecValTok{0}\NormalTok{);ridge\_model}
\end{Highlighting}
\end{Shaded}

\begin{verbatim}
## 
## Call:  cv.glmnet(x = x, y = y, alpha = 0) 
## 
## Measure: Mean-Squared Error 
## 
##     Lambda Index Measure     SE Nonzero
## min 0.6942   100   7.763 1.9951      15
## 1se 1.6037    91   9.503 0.9878      15
\end{verbatim}

\begin{Shaded}
\begin{Highlighting}[]
\NormalTok{best\_lambda }\OtherTok{\textless{}{-}}\NormalTok{ ridge\_model}\SpecialCharTok{$}\NormalTok{lambda.min}
\NormalTok{best\_lambda}
\end{Highlighting}
\end{Shaded}

\begin{verbatim}
## [1] 0.6941839
\end{verbatim}

\begin{Shaded}
\begin{Highlighting}[]
\NormalTok{best\_model }\OtherTok{\textless{}{-}} \FunctionTok{glmnet}\NormalTok{(x, y, }\AttributeTok{alpha =} \DecValTok{0}\NormalTok{,}\AttributeTok{lambda =}\NormalTok{ best\_lambda);best\_model}
\end{Highlighting}
\end{Shaded}

\begin{verbatim}
## 
## Call:  glmnet(x = x, y = y, alpha = 0, lambda = best_lambda) 
## 
##   Df  %Dev Lambda
## 1 15 92.82 0.6942
\end{verbatim}

\begin{Shaded}
\begin{Highlighting}[]
\FunctionTok{coef}\NormalTok{(best\_model, }\AttributeTok{s =} \StringTok{"lambda.min"}\NormalTok{)}
\end{Highlighting}
\end{Shaded}

\begin{verbatim}
## 16 x 1 sparse Matrix of class "dgCMatrix"
##                      s1
## (Intercept) 19.18924478
## age          0.38880181
## weight       2.30750495
## height       0.52787998
## adipos       0.46286715
## free        -6.43075409
## neck         0.09936367
## chest        1.27848771
## abdom        3.19204398
## hip          1.03998826
## thigh        1.01855289
## knee         0.72980876
## ankle        0.23945778
## biceps       0.47686916
## forearm      0.52888997
## wrist       -0.33165866
\end{verbatim}

\begin{Shaded}
\begin{Highlighting}[]
\FunctionTok{plot}\NormalTok{(ridge\_model)}
\end{Highlighting}
\end{Shaded}

\includegraphics{Hw4_126_files/figure-latex/unnamed-chunk-10-1.pdf}

\end{document}
